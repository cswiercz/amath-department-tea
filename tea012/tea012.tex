\documentclass[12pt]{article}

\usepackage{graphicx}
\usepackage{tikz}
\usepackage{array}

\title{AMath Tea Time --- Puzzle \#12}
\author{}
\date{\vspace{-1cm}1 December 2015}

\begin{document}

\maketitle
\pagenumbering{gobble}

\subsection*{Problem}

\noindent A ``number bracelet'' is a sequence of numbers (or a fashionable
mathematical accessory) constructed in the following way: first, choose two
integers between zero and nine. The next number of the sequence is the {\bf ones
  digit} of the sum of the previous two elements of the sequence. Continue to
write out this sequence until you arrive at the same pair of numbers as the {\bf
  initial pair}. \\

\noindent For example, if the first two elements of the number bracelet are 2
and 6 the resulting bracelet is,

\[
2,6,8,4,2,6.
\]

\noindent If, instead, the bracelet begins with 1 and 3 the result is,

\[
1,3,4,7,1,8,9,7,6,3,9,2,1,3.
\]

\begin{itemize}
  \item[(a)] Does every number bracelet terminate?
  \item[(b)] Your wise-ass friend instead begins their number bracelet with two
    {\bf rational} numbers between 0 and 10. Will their bracelet terminate?
  \item[(c)] What if they choose any two {\bf real} numbers between 0 and 10?
\end{itemize}

\subsection*{Hints}

{
\par\vspace*{\fill}
\noindent \small \it
If you have any puzzles to share then send them my way at {\tt
  cswiercz@uw.edu}!
}

\end{document}
